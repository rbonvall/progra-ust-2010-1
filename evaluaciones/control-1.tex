\documentclass[12pt,spanish]{article}
\usepackage[utf8]{inputenc}
\usepackage{babel}
\usepackage{fullpage}
\usepackage{url}
\usepackage{palatino}

\newcommand{\respuesta}{%
  \framebox[0.9\textwidth]{\rule[6.2ex]{0pt}{0pt}}%
  %\fbox{\vfill}%
}
\newcommand{\nombre}{%
  \framebox[0.8\textwidth]{\rule[2.3ex]{0pt}{0pt}}%
}

\begin{document}
  \thispagestyle{empty}
  {\Large\bfseries Programación---Control 1, viernes 16 de abril de 2010}

  Nombre: \nombre

  \begin{enumerate}
    \item ¿Qué es una expresión? \\
      \respuesta
    \item ¿Qué es la precedencia de operadores? \\
      \respuesta
    \item ¿Cómo puede uno cambar la precedencia de operadores en un programa? \\
      \respuesta
    \item ¿Qué es un error de sintaxis? \\
      \respuesta
    \item ¿Qué hace el operador módulo (\verb+%+)? \\
      \respuesta
    \item ¿Cuál es la diferencia entre los operadores \verb+/+ y \verb+//+? \\
      \respuesta
    \item ¿Qué tipo de datos tiene el resultado de la operación \verb+==+? \\
      \respuesta
    \item ¿Qué tipo de datos tiene el resultado de la función \verb+input()+? \\
      \respuesta
    \item ¿Qué hace la función \verb+print()+? \\
      \respuesta
  \end{enumerate}
\end{document}

