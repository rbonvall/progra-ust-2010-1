\documentclass[10pt,spanish]{article}
\usepackage[utf8]{inputenc}
\usepackage{babel}
\usepackage{fullpage}
\usepackage{url}
\usepackage{listings}
\usepackage{multirow}
\usepackage{mathpazo}
\usepackage{courier}
\usepackage{xcolor}
\usepackage{textcomp}
\usepackage{pifont}

\newcommand{\onelinerule}{\rule[2.3ex]{0pt}{0pt}}
\newcommand{\twolinerule}{\rule[6.2ex]{0pt}{0pt}}
\newcommand{\respuesta}{\framebox[\textwidth]{\twolinerule}}
\newcommand{\nombre}{\framebox[0.8\textwidth]{\onelinerule}}
\renewcommand{\arraystretch}{1.2}
\newcommand{\fixin}{\lstinline[columns=fixed]}
\newcommand{\pond}[1]%{{}[#1\%]}
                     {}

\newcommand{\T}{\ding{51}}
%\newcommand{\F}{\ding{55}}
\newcommand{\F}{}

\lstdefinelanguage{py3}{%
  classoffset=0,%
    morekeywords={%
      False,class,finally,is,return,None,continue,for,lambda,try,%
      True,def,from,nonlocal,while,and,del,global,not,with,%
      as,elif,if,or,yield,assert,else,import,pass,break,except,in,raise},%
  classoffset=1,
    morekeywords={int,float,str,abs,len,input,print,exit,range,min,max},
    keywordstyle=\color{black!50}\bfseries,%
  classoffset=0,%
  sensitive=true,%
  morecomment=[l]\#,%
  morestring=[b]',%
  morestring=[b]",%
  stringstyle=\em,%
}
\lstset{language=py3}
\lstset{basicstyle=\ttfamily}
\lstset{columns=fixed}
\lstset{upquote=true}

\begin{document}
  \thispagestyle{empty}
  \pagestyle{empty}
  {\Large\bfseries Programación---Prueba solemne 2, miércoles 19 de mayo de 2010}

  Nombre: \nombre

  \begin{enumerate}
%    \item
%      Escriba una función \lstinline+ordenados(L)+
%      que revise que los elementos de la lista \verb+L+
%      están ordenados de menor a mayor:
%      \begin{lstlisting}
%>>> ordenados([6, 1, 7, 4])
%False
%>>> ordenados([1, 4, 6, 7])
%True
%      \end{lstlisting}
%
%      \framebox[\textwidth]{\rule[20.0ex]{0pt}{0pt}}
%
    \item
      En el siguiente programa,
      indique cuáles son las variables locales
      y cuáles son las variables globales.
      \begin{minipage}{5em}
          \begin{lstlisting}
a = 6
b = True
def f(c):
    d = 3
    return 19 * d
e = f(a)
print(e)
          \end{lstlisting}
      \end{minipage}

      Variables locales:

      \framebox[\textwidth]{\rule[5.0ex]{0pt}{0pt}}

      Variables globales:

      \framebox[\textwidth]{\rule[5.0ex]{0pt}{0pt}}

    \item
      Escriba una función \lstinline+producto(valores)+
      que entregue el producto de los elementos
      de la lista \lstinline+valores+.

    \newpage
    \item
      La asistencia de los alumnos a clases
      puede ser llevada en una tabla como la siguiente:

      \begin{tabular}{|l|c|c|c|c|c|c|c|}\hline
        Clase    & 1& 2& 3& 4& 5& 6& 7\\\hline\hline
        Pepito   &\T&\T&\T&\F&\F&\F&\F\\\hline
        Yayita   &\T&\T&\T&\F&\T&\F&\T\\\hline
        Fulanita &\T&\T&\T&\T&\T&\T&\T\\\hline
        Panchito &\T&\T&\T&\F&\T&\T&\T\\\hline
      \end{tabular}

      En un programa,
      esta información puede ser representada
      usando listas:
      \begin{lstlisting}
>>> alumnos = ['Pepito', 'Yayita', 'Fulanita', 'Panchito']
>>> asistencia = [
...     [True,  True,  True,  False, False, False, False],
...     [True,  True,  True,  False, True,  False, True ],
...     [True,  True,  True,  True,  True,  True,  True ],
...     [True,  True,  True,  False, True,  True,  True ]]
      \end{lstlisting}

      \begin{enumerate}
        \item
          Escriba una función \lstinline+total_por_alumno(asistencia)+
          que entregue una lista
          con el número de clases asistidas por cada alumno:
          \begin{lstlisting}
>>> total_por_alumno(asistencia)
[3, 5, 7, 6]
          \end{lstlisting}
        \item
          Escriba una función \lstinline+alumno_estrella(asistencia)+
          que indique qué alumno asistió a más clases:
          \begin{lstlisting}
>>> alumno_estrella(asistencia)
'Fulanita'
          \end{lstlisting}
      \end{enumerate}
      \framebox[\textwidth]{\rule[55.0ex]{0pt}{0pt}}

  \end{enumerate}
\end{document}

