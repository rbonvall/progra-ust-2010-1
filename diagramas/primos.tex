\documentclass{minimal}
\usepackage[utf8]{inputenc}
\usepackage[spanish]{babel}
\usepackage{mathpazo}
\usepackage{amsmath}
\usepackage{tikz}
\usetikzlibrary{calc,shapes,arrows}

\newcommand{\assign}{\ensuremath{:=}}
\newcommand{\var}[1]{\ensuremath{\text{\texttt{#1}}}}

\begin{document}
\pagestyle{empty}
\tikzstyle{decision} = [
  diamond,
  very thick,
  draw=red!50!black!50,
  %fill=red!20, 
  aspect=2,
  %text badly centered,
  top color=white,
  bottom color=red!50!black!20,
]
\tikzstyle{stmt} = [
  rectangle,
  very thick,
  draw=blue!50!black!50,
  %fill=blue!20, 
  text centered,
  minimum height=5ex,
  top color=white,
  bottom color=blue!50!black!20,
]
\tikzstyle{node} = [
  circle,
  very thick,
  draw=orange!50!black!50,
  fill=orange!20,
  minimum size=7ex,
]
\tikzstyle{io} = [
  very thick,
  draw=green!50!black!50,
  trapezium,
  trapezium left angle=80,
  trapezium right angle=-80,
  %fill=green!20!black!10,
  %rounded corners,
  %minimum height=5ex,
  top color=white,
  bottom color=green!50!black!20,
  text centered,
]
\tikzstyle{conn} = [very thick, draw=black!50, -latex']



\begin{tikzpicture}[node distance=13ex, auto]
    % Place nodes
    \node [node] (start) { inicio };
    \node [io, below of=start, yshift=1em] (input) { $n = \var{entrada}()$ };
    \node [stmt, below of=input, yshift=1em] (init-d) { $d = 2$ };
    \node [decision, below of=init-d] (terminate) { ¿$d\le n/2$? };
    \node [decision, below of=terminate] (divides) { ¿$n\bmod d = 0$? };
    \node [stmt, right of=divides, node distance=13em] (incr) { $d = d + 1$ };
    \node [io, below of=divides] (print-composite) {
        $\var{escribir}(n, \text{``es compuesto''})$
    };
    \node [io, left of=print-composite,node distance=15em] (print-prime) {
        $\var{escribir}(n, \text{``es primo''})$
    };
    \node [node, below of=print-composite] (end) {fin};

    \path [conn] (start) -- (input);
    \path [conn] (input) -- (init-d);
    \path [conn] (init-d) -- (terminate);
    \path [conn] (terminate) --
                 node [very near start] {sí}
                 (divides);
    \path [conn] (terminate.west) -|
                 node [very near start] {no}
                 (print-prime.north);
    \path [conn] (divides) --
                 node [very near start] {sí}
                 (print-composite);
    \path [conn] (divides) --
                 node [very near start] {no}
                 (incr);
    \path [conn] (incr.north) |- (terminate.east);
    \node [inner sep=0pt] (m) at ($ (print-composite.south)!.5!(end.north) $) {};
    \path [conn] (print-prime.south) |- (m);
    \path [conn] (print-composite) -- (end);
\end{tikzpicture}

\end{document}
