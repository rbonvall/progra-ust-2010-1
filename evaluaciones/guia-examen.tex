\documentclass[10pt,spanish]{article}
\usepackage[utf8]{inputenc}
\usepackage{babel}
\usepackage{fullpage}
\usepackage{listings}
\usepackage{mathpazo}
\usepackage{courier}
\usepackage{xcolor}
\usepackage{textcomp}
\usepackage{amssymb}
\usepackage{tikz}

\newcommand{\onelinerule}{\rule[2.3ex]{0pt}{0pt}}
\newcommand{\twolinerule}{\rule[6.2ex]{0pt}{0pt}}
\newcommand{\respuesta}{\framebox[\textwidth]{\twolinerule}}
\newcommand{\nombre}{\framebox[0.9\textwidth]{\onelinerule}}

\lstdefinelanguage{py3}{%
  classoffset=0,%
    morekeywords={%
      False,class,finally,is,return,None,continue,for,lambda,try,%
      True,def,from,nonlocal,while,and,del,global,not,with,%
      as,elif,if,or,yield,assert,else,import,pass,break,except,in,raise},%
    keywordstyle=\color{black!80}\bfseries,%
  classoffset=1,
    morekeywords={int,float,str,abs,len,input,print,exit,range,min,max,%
      set,dict,tuple,list,type},
    keywordstyle=\color{black!50}\bfseries,%
  classoffset=0,%
  sensitive=true,%
  morecomment=[l]\#,%
  morestring=[b]',%
  morestring=[b]",%
  stringstyle=\em,%
}
\lstset{language=py3}
\lstset{basicstyle=\ttfamily}
\lstset{columns=fixed}
\lstset{upquote=true}
\lstset{upquote=true}
\lstset{showstringspaces=false}

\begin{document}
  \thispagestyle{empty}
  \pagestyle{empty}
  {\Large\bfseries Programación---Guía de preparación para el examen}

  \vspace{1ex}
  \begin{enumerate}
    \item
      Escriba la salida de cada programa.

      \foreach \x in {1,2,3} {
        \noindent
        \begin{minipage}[b]{15.5em}
          \lstinputlisting{programas-guia/p\x.py}
          \framebox[14em]{\rule[5ex]{0pt}{0pt}}
          \vspace{0.7em}
        \end{minipage}
      }

    \item
      Escriba una función \lstinline!es_divisible(n, d)!
      que indique si \(n\) es divisible por \(d\):
\begin{lstlisting}
>>> es_divisible(15, 5)
True
>>> es_divisible(15, 6)
False
\end{lstlisting}
 
    \item
      Escriba una función \lstinline!es_primo(n)!
      que indique si el número \lstinline!n! es primo o no:
\begin{lstlisting}
>>> es_primo(9)                  >>> es_primo(2)         
False                            True                    
>>> es_primo(7)                  >>> es_primo(1)         
True                             False                   
\end{lstlisting}

    \item
      Escriba una función \lstinline!digitos(n)!
      que entregue una lista con los dígitos de \lstinline!n!:
\begin{lstlisting}
>>> digitos(142857)
[1, 4, 2, 8, 5, 7]
\end{lstlisting}

    \item
      Escriba las siguientes funciones que \emph{cuentan cosas}.
      Todas las funciones reciben como parámetro una lista \verb+L+
      de números enteros, y retornan el resultado indicado.
      \begin{enumerate}
        \item \lstinline!cuenta(L)!:
          cuenta cuántos elementos tiene \verb+L+.
        \item \lstinline!cuenta_i(L)!:
          cuenta cuántos elementos impares tiene \verb+L+.
        \item \lstinline!cuenta_p(L)!:
          cuenta cuántos elementos primos tiene \verb+L+.
        \item \lstinline!cuenta_mp(L)!:
          cuenta cuántos elementos de \verb+L+
          son menores que el promedio.
        \item \lstinline!cuenta_m(L)!:
          cuenta cuántos elementos de \verb+L+ terminan en 7.
      \end{enumerate}
      ¿Cuántos de los números primos entre 1 y 1000 terminan en 7?
      (Respuesta: 46).

    \item
      Escriba las siguientes funciones que \emph{suman cosas}.
      Todas las funciones reciben como parámetro una lista \verb+L+
      de números enteros, y retornan el resultado indicado.
      \begin{enumerate}
        \item \lstinline!suma(L)!:
          suma los elementos de \verb+L+.
        \item \lstinline!suma_i(L)!:
          suma los elementos impares de \verb+L+.
        \item \lstinline!suma_c(L)!:
          suma los cuadrados de los elementos de \verb+L+.
        \item \lstinline!suma_ci(L)!:
          suma los cuadrados de los elementos impares de \verb+L+.
      \end{enumerate}
      ¿Cuánto es la suma de los cuadrados de los números primos entre 1 y 1000?
      (Respuesta: 49.345.379).

    \item
      Escriba las siguientes funciones que \emph{mapean} listas,
      es decir, convierten una lista en otra, elemento por elemento.
      Todas las funciones reciben como parámetro una lista \verb+L+
      de números enteros, y retornan una nueva lista.
      \begin{enumerate}
        \item \lstinline!cuadrados(L)!:
          entrega todos los elementos de \verb+L+ elevados al cuadrado.
        \item \lstinline!ultimos_digitos(L)!:
          entrega el último dígito de cada valor de \verb+L+.
        \item \lstinline!son_primos(L)!:
          entrega una lista con valores \lstinline+True+ y \lstinline+False+,
          dependiendo si el elemento respectivo de \verb+L+ es primo o no:
\begin{lstlisting}
>>> son_primos([91, 77, 31, 29, 27])
[False, False, True, True, False]
\end{lstlisting}
      \end{enumerate}

    \item
      Escriba las siguientes funciones que \emph{reducen} listas,
      es decir, obtienen un único valor a partir de todos los elementos de una lista.
      Todas las funciones reciben como parámetro una lista \verb+L+
      de números enteros, y retornan un único valor.
      \begin{enumerate}
        \item \lstinline!producto(L)!:
          entrega el producto de los elementos de \verb+L+.
        \item \lstinline!maximo(L)!:
          entrega el mayor de los elementos de \verb+L+.
          (Prohibido usar la función \lstinline+max+).
        \item \lstinline!hay_multiplo_de_siete(L)!:
          retorna \lstinline+True+ si alguno de los valores de \verb+L+
          es un múltiplo de 7, y \lstinline+False+ si ninguno lo es.
      \end{enumerate}

    \item
      Escriba las siguientes funciones que \emph{filtran} listas,
      es decir, sólo conservan los elementos de una lista
      que cumplen una condición.
      Todas las funciones reciben como parámetro una lista \verb+L+
      de números enteros, y retornan una nueva lista.
      \begin{enumerate}
        \item \lstinline!positivos(L)!: 
          entrega una lista con los valores de \verb+L+ que son positivos.
        \item \lstinline!mayores_que(L, n)!: 
          entrega una lista con los valores de \verb+L+ que son mayores que \verb+n+.
        \item \lstinline!primos(L)!:
          entrega una lista con los valores de \verb+L+ que son primos.
        \item \lstinline!mayores_que_el_promedio(L)!:
          entrega una lista con los valores de \verb+L+
          que son mayores que el promedio.
        \item \lstinline!con_siete(L)!:
          entrega una lista con los valores de \verb+L+
          que tienen el dígito 7.
      \end{enumerate}
      ¿Cuál es el producto de todos los números primos menores que 100
      que tienen algún dígito 7? (Respuesta: \(
      7\cdot 17\cdot 37\cdot 47\cdot 67\cdot 71\cdot 73\cdot 79\cdot 97
      =\) 550.682.633.299.463)

    \item
      Un \emph{cubo perfecto} es un entero
      que es igual a otro entero elevado a tres.
      Los primeros cubos perfectos son 1, 8, 27 y 64.
      \begin{enumerate}
        \item Escriba una función \lstinline!es_cubo_perfecto(n)!
          que indique si un número es o no un cubo perfecto:
\begin{lstlisting}
>>> es_cubo_perfecto(27)
True
>>> es_cubo_perfecto(49)
False
\end{lstlisting}
        \item Escriba una función \lstinline!es_suma_de_cubos_perfectos(n)!
          que indique si un número puede ser expresado
          como la suma de dos cubos perfectos.
          Por ejemplo, el 35 sí puede (\(2^3 + 3^3\)),
          pero el 41 no.
\begin{lstlisting}
>>> es_suma_de_cubos_perfectos(35)
True
>>> es_suma_de_cubos_perfectos(41)
False
\end{lstlisting}
        \item (Difícil) El \emph{número de Hardy-Ramanujan}
          es el menor número entero que puede ser expresado
          como la suma de dos cubos perfectos
          de dos maneras distintas (\(a^3 + b^3 = c^3 + d^3\)).
          Escriba un programa que descubra este número.
      \end{enumerate}

    \item
      Considere el siguiente trozo de programa:
\begin{lstlisting}
d = {
  (1, 2): [{1, 2}, {3}, {1, 3}],
  (2, 1): [{3}, {1, 2}, {1, 2, 3}],
  (2, 2): [{}, {2, 3}, {1, 3}],
}
\end{lstlisting}
      Indique el valor y el tipo de las siguientes expresiones:
      \begin{itemize}
        \item \lstinline!d[(1, 2)][2]!
          \hfill
          [Respuesta: el valor es \lstinline!{1, 3}!,
          el tipo es \lstinline!set! (conjunto)]
        \item \lstinline!len(d)!
          \hfill
          [Respuesta: el valor es \lstinline!3!,
          el tipo es \lstinline!int! (entero)]
        \item \lstinline!(1, 2)!
        \item \lstinline!(1, 2)[1]!
        \item \lstinline!d[(1, 2)][1]!
        \item \lstinline!d[(1, 2)]!
        \item \lstinline!len(d[(2, 1)])!
        \item \lstinline!len(d[(2, 1)][1])!
        \item \lstinline!d[(2, 2)][1] & d[(1, 2)][2]!
        \item \lstinline!(d[(2, 2)] + d[(2, 1)])[4]!
      \end{itemize}

      Puede verificar su respuesta en el intérprete interactivo.
      Para ver el tipo, puede usar la función \lstinline+type+:
\begin{lstlisting}
>>> v = d[(1, 2)][2]
>>> v
{1, 3}
>>> type(v)
<class 'set'>
\end{lstlisting}

    \item Ejercicios para llenar diccionarios.
      \begin{enumerate}
        \item 
          Escriba una función \lstinline+largo_palabras(oracion)+
          que asocie a cada palabra de la oración
          el total de letras que tiene:
\begin{lstlisting}
>>> largo_palabras('el gato y el pato son amigos')
{'el': 2, 'son': 3, 'gato': 4, 'y': 1, 'amigos': 6, 'pato': 4}
\end{lstlisting}
        \item 
          Escriba una función \lstinline+contar_vocales(oracion)+
          que asocie a cada vocal
          la cantidad de veces que aparece en la oración:
\begin{lstlisting}
>>> contar_vocales('el gato y el pato son amigos')
{'a': 3, 'i': 1, 'e': 2, 'u': 0, 'o': 4}
\end{lstlisting}
      \end{enumerate}

    \item
      Las fechas de nacimiento de varias personas
      están guardadas en un diccionario.
      Cada llave es el nombre de una persona,
      y cada valor es una tupla de tres elementos:
      el año, el mes y el día.
\begin{lstlisting}
fechas_de_nacimiento = {
  'Pepito': (1983, 10, 27),
  'Fulanita': (1962, 2, 14),
  'Jaimito': (1955, 3, 3),
  'Yayita': (1990, 12, 4),
}
\end{lstlisting}
      \begin{enumerate}
        \item Escriba una función \lstinline!edad(p)!
          que entregue la edad de la persona \verb+p+.
        \item Escriba una función \lstinline!mas_viejo()!
          que indique quién es la persona más vieja.
        \item Escriba una función \lstinline!primer_cumple()!
          que indique la fecha del primer cumpleaños del año.
      \end{enumerate}

    \item
      Todo número entero mayor que uno
      puede ser expresado como un único producto de números primos.
      Esto se llama la \emph{factorización prima} del número.
      Por ejemplo:
      \(84 = 2\cdot 2\cdot 3\cdot 7\),
      \(39 = 3\cdot 13\), y
      \(539 = 7\cdot 7\cdot 11\).
      \begin{enumerate}
        \item Escriba una función \lstinline!fp(n)!
          que entregue la lista de los factores primos de \lstinline!n!:
\begin{lstlisting}
>>> fp(84)
[2, 2, 3, 7]
>>> fp(39)
[3, 13]
>>> fp(7)
[7]
\end{lstlisting}
          Puede comprobar que el resultado es correcto
          usando la función \lstinline+producto+
          que usted mismo escribió para un ejercicio anterior:
\begin{lstlisting}
>>> producto(fp(84))
84
>>> producto(fp(12345))
12345
\end{lstlisting}
        \item Escriba una función \lstinline!factorizar_valores(L)!
          que entregue un diccionario
          asociando a cada valor de la lista \lstinline!L!
          su factorización prima:
\begin{lstlisting}
>>> factorizar_valores([84, 539, 12])
{12: [2, 2, 3], 539: [7, 7, 11], 84: [2, 2, 3, 7]}
\end{lstlisting}
      \end{enumerate}
      Solución de la parte \emph{a})
      (intente resolverla por su cuenta antes de mirarla):
\begin{lstlisting}
def fp(n):
    f = []
    while n > 1:
        d = menor_factor_primo(n)
        f.append(d)
        n = n // d
    return f
\end{lstlisting}
Definir la función \lstinline!menor_factor_primo(n)!
queda de tarea.

  \end{enumerate}
\end{document}

