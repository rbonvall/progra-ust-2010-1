\documentclass[12pt,spanish]{article}
\usepackage[utf8]{inputenc}
\usepackage{babel}
\usepackage{fullpage}
\usepackage{url}
\usepackage{mathpazo}

\newcommand{\respuesta}{%
  \framebox[0.9\textwidth]{\rule[6.2ex]{0pt}{0pt}}%
  %\fbox{\vfill}%
}
\newcommand{\nombre}{%
  \framebox[0.8\textwidth]{\rule[2.3ex]{0pt}{0pt}}%
}

\begin{document}
  \thispagestyle{empty}
  {\Large\bfseries Programación---Control 2, viernes 14 de mayo de 2010}

  Nombre: \nombre
  \vspace{2em}

  La \emph{media geométrica} de una secuencia de $n$ números reales
  $x_1, x_2, \dots, x_n$ se define como:
  \[
    G = \left(x_1\cdot x_2\cdot\;\cdots\;\cdot x_n\right)^{1/n}.
  \]
  
  \begin{enumerate}
    \item Escriba una función \texttt{mg(L)},
      que retorne la media geométrica
      de los valores que están en la lista \texttt{L}.

      \framebox[\textwidth]{\rule[25ex]{0pt}{0pt}}
    \item Usando la función \texttt{mg},
      escriba un programa que:
      \begin{itemize}
        \item pregunte al usuario cuántos valores ingresará ($n$);
        \item pida al usuario que ingrese cada uno de los $n$ valores;
        \item imprima la media geométrica de los valores.
      \end{itemize}

      \framebox[\textwidth]{\rule[25ex]{0pt}{0pt}}



  \end{enumerate}
\end{document}

