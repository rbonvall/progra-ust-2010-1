\documentclass[10pt,spanish]{article}
\usepackage[utf8]{inputenc}
\usepackage{babel}
\usepackage{fullpage}
\usepackage{listings}
\usepackage{mathpazo}
\usepackage{courier}
\usepackage{xcolor}
\usepackage{textcomp}
\usepackage{amssymb}
\usepackage{tikz}

\newcommand{\onelinerule}{\rule[2.3ex]{0pt}{0pt}}
\newcommand{\twolinerule}{\rule[6.2ex]{0pt}{0pt}}
\newcommand{\respuesta}{\framebox[\textwidth]{\twolinerule}}
\newcommand{\nombre}{\framebox[0.8\textwidth]{\onelinerule}}
%\renewcommand{\arraystretch}{1.2}
\newcommand{\cv}{\texttt}

\newcommand{\card}[3]{%
  \draw #1 rectangle +(2, 4);
  %\draw +(0.5, 3.5) node {#2};
  %\draw +(0.5, 2.6) node {\ensuremath{#3}};
  \draw +(-1.5, -0.5) node {#2};
  \draw +(0.5, 2.6) node {\ensuremath{#3}};
}

\lstdefinelanguage{py3}{%
  classoffset=0,%
    morekeywords={%
      False,class,finally,is,return,None,continue,for,lambda,try,%
      True,def,from,nonlocal,while,and,del,global,not,with,%
      as,elif,if,or,yield,assert,else,import,pass,break,except,in,raise},%
    keywordstyle=\color{black!80}\bfseries,%
  classoffset=1,
    morekeywords={int,float,str,abs,len,input,print,exit,range,min,max,%
      set,dict,tuple,list},
    keywordstyle=\color{black!50}\bfseries,%
  classoffset=0,%
  sensitive=true,%
  morecomment=[l]\#,%
  morestring=[b]',%
  morestring=[b]",%
  stringstyle=\em,%
}
\lstset{language=py3}
\lstset{basicstyle=\ttfamily}
\lstset{columns=fixed}
\lstset{upquote=true}
\lstset{upquote=true}
\lstset{showstringspaces=false}

\begin{document}
  \thispagestyle{empty}
  \pagestyle{empty}
  {\Large\bfseries Programación---Prueba solemne 3, viernes 18 de junio de 2010}

  Nombre: \nombre

  \begin{enumerate}
    \item
      Escriba la salida de cada programa.

      \foreach \x in {1,2,...,6} {
        \noindent
        \begin{minipage}[b]{15.5em}
          \lstinputlisting{programas-prueba-3/p\x.py}
          \framebox[14em]{\rule[10ex]{0pt}{0pt}}
          \vspace{0.7em}
        \end{minipage}
      }

    \item
      Escriba una función \lstinline!contar_iniciales(oracion)!
      que retorne un diccionario cuyas llaves sean letras,
      y que el valor asociado indique
      cuántas palabras de la oración comienzan con la letra:
          \begin{lstlisting}
>>> contar_iniciales('El elefante avanza hacia Asia')
{'e': 2, 'h': 1, 'a': 2}
>>> contar_iniciales('Varias vacas vuelan sobre Venezuela')
{'s': 1', 'v': 4}
          \end{lstlisting}
      
      \framebox[\textwidth]{\rule[40.0ex]{0pt}{0pt}}

    \newpage
    \item
      Las palabras \emph{panvocálicas}
      son las que tienen las cinco vocales.
      Por ejemplo:
      c\underline{e}ntr\underline{i}f\underline{u}g\underline{a}d\underline{o},
      b\underline{i}s\underline{a}b\underline{u}\underline{e}l\underline{o},
      h\underline{i}p\underline{o}t\underline{e}n\underline{u}s\underline{a}.

      \begin{enumerate}

        \item
          Escriba una función \lstinline!es_panvocalica(palabra)!
          que indique si una palabra es panvocálica o no:
          \begin{lstlisting}
>>> es_panvocalica('educativo')
True
>>> es_panvocalica('pedagogico')
False
          \end{lstlisting}

          \framebox[0.9\textwidth]{\rule[35.0ex]{0pt}{0pt}}
          \vspace{1em}

        \item
          Escriba una función \lstinline!cuenta_pv(oracion)!
          que cuente cuántas palabras panvocálicas tiene una oración:
          \begin{lstlisting}
>>> cuenta_pv('el abuelito mordisquea el aceituno con contundencia')
4
>>> cuenta_pv('la cincuentona estudiosa va a casa')
2
>>> cuenta_pv('los hipopotamos bailan al amanecer')
0
          \end{lstlisting}

          \framebox[0.9\textwidth]{\rule[35.0ex]{0pt}{0pt}}

      \end{enumerate}


    \newpage
    \item
      En los juegos de naipes, una carta tiene dos atributos:
      un valor (\cv{A}, \cv{2}, \cv{3}, \cv{4}, \cv{5}, \cv{6}, \cv{7}, \cv{8},
                \cv{9}, \cv{10}, \cv{J}, \cv{Q} o \cv{K})
      y una pinta (\(\heartsuit\), \(\spadesuit\), \(\diamondsuit\) o \(\clubsuit\)).

      En un programa,
      una carta puede ser representada como una tupla de dos elementos:
      el valor y la pinta.
      El valor es un número del 1 al 13,
      y la pinta es un string
      (\lstinline!'C'!, \lstinline!'P'!, \lstinline!'D'! o \lstinline!'T'!).

      Una mano puede ser representada como un conjunto de cartas.
      Por ejemplo, uno puede representar la mano
      \fbox{\cv{5}\(\clubsuit\)}
      \fbox{\cv{2}\(\heartsuit\)}
      \fbox{\cv{1}\(\spadesuit\)}
      \fbox{\cv{Q}\(\heartsuit\)}
      \fbox{\cv{K}\(\clubsuit\)}
      de la siguiente manera:
      \begin{lstlisting}
mano = {(5, 'T'), (2, 'C'), (1, 'P'), (12, 'C'), (13, 'T')}
      \end{lstlisting}

      En el carioca (un juego de naipes chileno)
      una \emph{escala} es una mano de cuatro cartas
      que tienen la misma pinta y que tienen valores consecutivos.
      Por ejemplo:
      \begin{itemize}
        \item
          \fbox{\cv{3}\(\heartsuit\)}
          \fbox{\cv{6}\(\heartsuit\)}
          \fbox{\cv{5}\(\heartsuit\)}
          \fbox{\cv{4}\(\heartsuit\)}
          es una escala, ya que todas las cartas tienen pinta \(\heartsuit\)
          y valores consecutivos desde el~\cv{3} hasta el~\cv{6}.

        \item
          \fbox{\cv{3}\(\clubsuit\)}
          \fbox{\cv{6}\(\diamondsuit\)}
          \fbox{\cv{5}\(\diamondsuit\)}
          \fbox{\cv{4}\(\heartsuit\)}
          no es una escala, ya que las cartas tienen pintas diferentes.

        \item
          \fbox{\cv{3}\(\clubsuit\)}
          \fbox{\cv{A}\(\clubsuit\)}
          \fbox{\cv{J}\(\clubsuit\)}
          \fbox{\cv{5}\(\clubsuit\)}
          no es una escala, ya que los valores no son consecutivos.

        \item
          \fbox{\cv{3}\(\spadesuit\)}
          \fbox{\cv{4}\(\spadesuit\)}
          \fbox{\cv{5}\(\spadesuit\)}
          no es una escala, ya que la mano no tiene cuatro cartas.
      \end{itemize}

      Escriba una función \lstinline!es_escala(mano)!
      que indique si la mano es una escala o no:
      \begin{lstlisting}
>>> es_escala({(3, 'C'), (6, 'C'), (5, 'C'), (4, 'C')})
True
>>> es_escala({(3, 'T'), (6, 'D'), (5, 'D'), (4, 'C')})
False
>>> es_escala({(3, 'T'), (1, 'T'), (11, 'T'), (5, 'T'))
False
>>> es_escala({(3, 'C'), (4, 'C'), (5, 'C')})
False
      \end{lstlisting}

      \framebox[\textwidth]{\rule[50.0ex]{0pt}{0pt}}







  \end{enumerate}
\end{document}

