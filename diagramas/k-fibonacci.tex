\documentclass{minimal}
\usepackage[utf8]{inputenc}
\usepackage[spanish]{babel}
\usepackage{mathpazo}
\usepackage{amsmath}
\usepackage{tikz}
\usetikzlibrary{calc,shapes,arrows}

\newcommand{\assign}{\ensuremath{=}}
\newcommand{\var}[1]{\ensuremath{\text{\texttt{#1}}}}

\begin{document}
\pagestyle{empty}
\tikzstyle{decision} = [
  diamond,
  very thick,
  draw=red!50!black!50,
  %fill=red!20, 
  aspect=2,
  %text badly centered,
  top color=white,
  bottom color=red!50!black!20,
]
\tikzstyle{stmt} = [
  rectangle,
  very thick,
  draw=blue!50!black!50,
  %fill=blue!20, 
  text centered,
  minimum height=5ex,
  top color=white,
  bottom color=blue!50!black!20,
]
\tikzstyle{node} = [
  circle,
  very thick,
  draw=orange!50!black!50,
  fill=orange!20,
  minimum size=7ex,
]
\tikzstyle{io} = [
  very thick,
  draw=green!50!black!50,
  trapezium,
  trapezium left angle=80,
  trapezium right angle=-80,
  %fill=green!20!black!10,
  %rounded corners,
  %minimum height=5ex,
  top color=white,
  bottom color=green!50!black!20,
  text centered,
]
\tikzstyle{conn} = [very thick, draw=black!50, -latex']



\begin{tikzpicture}[node distance=13ex, auto]
    % Place nodes
    \node [node] (start) { inicio };
    \node [io, below of=start] (read) {
        $n = \var{int}\bigl(
                 \var{input}(\text{``¿Cuál es el número $n$?''})
             \bigr)$
    };
    \node [decision, below of=read] (is-zero) { ¿$n == 0$? };
    \node [stmt, below of=is-zero] (init) {
        \begin{minipage}{7em}
            $\var{actual}\assign 1$ \\
            $\var{anterior}\assign 0$ \\
            $\var{cuenta}\assign 1$
        \end{minipage}
    };
    \node [stmt, left of=init, node distance=10em] (zero) {
        $\var{actual}\assign 0$
    };
    \node [decision, below of=init, node distance=18ex] (is-kth) {
        ¿$\var{cuenta} < n$?
    };
    \node [stmt, right of=is-kth, node distance=17em] (update) {
        \begin{minipage}{13em}
            $\var{suma}\assign \var{actual} + \var{anterior}$ \\
            $\var{anterior}\assign \var{actual}$ \\
            $\var{actual}\assign \var{suma}$ \\
            $\var{cuenta}\assign \var{cuenta} + 1$
        \end{minipage}
    };
    \node [io, below of=is-kth, node distance=18ex] (write-kth) {
        \(\var{print}%
           (\text{``El $n$-ésimo número de Fibonacci es''}, \var{actual})%
        \)
    };
    \node [node, below of=write-kth] (end) { fin };

    \path [conn] (start) -- (read);
    \path [conn] (read) -- (is-zero);
    \path [conn] (is-zero) --
                 node[very near start] {no}
                 (init);
    \path [conn] (is-zero.west) -|
                 node[very near start] {sí}
                 (zero);
    \path [conn] (init) -- (is-kth);
    \path [conn] (update.north) |-
                 ($ (init.south)!.5!(is-kth.north) $);
    \path [conn] (zero.south) |-
                 ($ (is-kth.south)!.5!(write-kth.north) $);
    \path [conn] (is-kth) -- 
                 node[near start] {sí}
                 (update);
    \path [conn] (is-kth) -- node[very near start] {no} (write-kth);
%    \path [conn] (is-kth.west) -| node[very near start] {sí} (write-kth.north);
    \path [conn] (write-kth)  -- (end);
\end{tikzpicture}

\end{document}
