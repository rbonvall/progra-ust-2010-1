\documentclass[10pt,spanish]{article}
\usepackage[utf8]{inputenc}
\usepackage{babel}
\usepackage{fullpage}
\usepackage{listings}
\usepackage{mathpazo}
\usepackage{courier}
\usepackage{xcolor}
\usepackage{textcomp}
\usepackage{amssymb}
\usepackage{tikz}

\newcommand{\onelinerule}{\rule[2.3ex]{0pt}{0pt}}
\newcommand{\twolinerule}{\rule[6.2ex]{0pt}{0pt}}
\newcommand{\respuesta}{\framebox[\textwidth]{\twolinerule}}
\newcommand{\nombre}{\framebox[0.9\textwidth]{\onelinerule}}

\lstdefinelanguage{py3}{%
  classoffset=0,%
    morekeywords={%
      False,class,finally,is,return,None,continue,for,lambda,try,%
      True,def,from,nonlocal,while,and,del,global,not,with,%
      as,elif,if,or,yield,assert,else,import,pass,break,except,in,raise},%
    keywordstyle=\color{black!80}\bfseries,%
  classoffset=1,
    morekeywords={int,float,str,abs,len,input,print,exit,range,min,max,%
      set,dict,tuple,list},
    keywordstyle=\color{black!50}\bfseries,%
  classoffset=0,%
  sensitive=true,%
  morecomment=[l]\#,%
  morestring=[b]',%
  morestring=[b]",%
  stringstyle=\em,%
}
\lstset{language=py3}
\lstset{basicstyle=\ttfamily}
\lstset{columns=fixed}
\lstset{upquote=true}
\lstset{upquote=true}
\lstset{showstringspaces=false}

\begin{document}
  \thispagestyle{empty}
  \pagestyle{empty}
  {\Large\bfseries Programación---Prueba recuperativa, lunes 19 de julio de 2010}

  Nombre: \nombre

  \vspace{1ex}
  \begin{enumerate}
    \item
      Escriba la salida de cada programa.

      \foreach \x in {1,2,...,9} {
        \noindent
        \begin{minipage}[b]{15.5em}
          \lstinputlisting{programas-prueba-rec/p\x.py}
          \framebox[14em]{\rule[5ex]{0pt}{0pt}}
          \vspace{0.7em}
        \end{minipage}
      }
    \item
      Rutee el siguiente programa con la entrada \lstinline!n = 56!,
      y determine qué es lo que hace el programa.

      \hspace{2em}
      \begin{minipage}{.4\textwidth}
        \lstinputlisting[frame=single,numbers=left,numberstyle=\small]{programas-prueba-rec/ruteo.py}
        \vspace{5ex}
        ¿Qué hace el programa?

        \framebox[\textwidth]{\rule[10.0ex]{0pt}{0pt}}
      \end{minipage}
      \hspace{4em}
      \begin{minipage}{.5\textwidth}
        \begin{tabular}{| p{2.5em} || *{4}{p{2.5em} |}}\hline
          Línea & \hfill\tt n & \hfill\tt r & \hfill\tt m \\\hline\hline
          \hfill\small 1 & \hfill 56 && \\\hline
          \hfill\small 2 & & \hfill 3 & \\\hline
          \hfill\small 3 &&& \\\hline
          \hfill\small 5 &&& \\\hline
          \hfill\small 6 &&& \\\hline
          &&& \\\hline &&& \\\hline &&& \\\hline
          &&& \\\hline &&& \\\hline &&& \\\hline
          &&& \\\hline &&& \\\hline &&& \\\hline
        \end{tabular}
      \end{minipage}

    \item
      Desarrolle un programa que pida al usuario
      que ingrese varios números, uno por uno.
      Cuando el usuario ingrese el 0,
      el programa debe terminar, entregando como salida
      la cantidad de números positivos y negativos que fueron ingresados.

      Por ejemplo,
      si el usuario ingresa los números
      \(5, -4, -2, -7, 7, -1\) y \(0\),
      el programa debe entregar la siguiente salida:%
      \begin{verbatim}
  Positivos: 2
  Negativos: 4
      \end{verbatim}
      \vspace{40ex}

    \item
      Dos palabras son \emph{anagramas}
      si tienen las mismas letras pero en otro orden.
      Por ejemplo, ``torpes'' y ``postre'' son anagramas,
      mientras que ``aparta'' y ``raptar'' no lo son,
      ya que ``raptar'' tiene una \emph{r} de más
      y una \emph{a} de menos.

      Escriba una función \lstinline!son_anagramas(p1, p2)!,
      que indique si las palabras \lstinline!p1! y \lstinline!p2!
      son anagramas:
      \begin{lstlisting}
>>> son_anagramas('torpes', 'postre')
True
>>> son_anagramas('aparta', 'raptar')
False
      \end{lstlisting}
      \newpage

    \item
      El diccionario \lstinline!paises! asocia cada persona
      con el conjunto de los países que ha visitado:
      \begin{lstlisting}
paises = {
  'Pepito': {'Chile', 'Argentina'},
  'Yayita': {'Francia', 'Suiza', 'Chile'},
  'John': {'Chile', 'Italia', 'Francia', 'Peru'},
  ...
}
\end{lstlisting}
      Escriba una función \lstinline!cuantos_en_comun(a, b)!,
      que indique cuántos países en común han visitado
      la persona \lstinline!a! y la persona \lstinline!b!:
      \begin{lstlisting}
>>> cuantos_en_comun('Pepito', 'John')
1
>>> cuantos_en_comun('John', 'Yayita')
2
      \end{lstlisting}


      \vspace{30ex}

    \item
      Las notas de un ramo están almacenadas
      en un archivo llamado \texttt{notas.txt}.
      Cada línea tiene el nombre del alumno y sus seis notas,
      separadas por dos puntos (``\verb!:!'').

      Escriba un programa que cree un nuevo archivo
      llamado \texttt{reporte.txt},
      en que cada línea indique
      si el alumno está aprobado (promedio \(\ge 4.0\))
      o reprobado (promedio \(< 4.0\)).

      \vspace{2ex}
      \begin{minipage}[c]{.45\textwidth}
        \texttt{notas.txt}:
        \begin{lstlisting}[language={},frame=single]
Pepito:3.3:3.7:3.7:6.7:1.1:2.5
Yayita:5.8:5.2:2.0:3.6:6.0:2.0
Fulanita:1.2:5.6:5.4:5.1:5.8:4.3
Moya:2.6:4.7:1.8:3.5:2.7:4.5
        \end{lstlisting}
      \end{minipage}
      \hspace{1em}
      \(\longrightarrow\)
      \hspace{1em}
      \begin{minipage}[c]{.3\textwidth}
        \texttt{reporte.txt}:
        \begin{lstlisting}[language={},frame=single]
Pepito reprobado
Yayita aprobado
Fulanita aprobado
Moya reprobado
        \end{lstlisting}
      \end{minipage}

  \end{enumerate}
\end{document}

