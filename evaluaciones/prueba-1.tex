\documentclass[11pt,spanish]{article}
\usepackage[utf8]{inputenc}
\usepackage{babel}
\usepackage{fullpage}
\usepackage{url}
\usepackage{listings}
\usepackage{multirow}
\usepackage{mathpazo}
\usepackage{courier}
\usepackage{xcolor}
\usepackage{textcomp}

\newcommand{\onelinerule}{\rule[2.3ex]{0pt}{0pt}}
\newcommand{\twolinerule}{\rule[6.2ex]{0pt}{0pt}}
\newcommand{\respuesta}{\framebox[\textwidth]{\twolinerule}}
\newcommand{\nombre}{\framebox[0.8\textwidth]{\onelinerule}}
\renewcommand{\arraystretch}{1.2}
\newcommand{\fixin}{\lstinline[columns=fixed]}
\newcommand{\pond}[1]%{{}[#1\%]}
                     {}

\lstdefinelanguage{py3}{%
  classoffset=0,%
    morekeywords={%
      False,class,finally,is,return,None,continue,for,lambda,try,%
      True,def,from,nonlocal,while,and,del,global,not,with,%
      as,elif,if,or,yield,assert,else,import,pass,break,except,in,raise},%
  classoffset=1,
    morekeywords={int,float,str,abs,len,input,print,exit,range,min,max},
    keywordstyle=\color{black!50}\bfseries,%
  classoffset=0,%
  sensitive=true,%
  morecomment=[l]\#,%
  morestring=[b]',%
  morestring=[b]",%
  stringstyle=\em,%
}
\lstset{language=py3}
\lstset{basicstyle=\ttfamily}
\lstset{columns=fixed}
\lstset{upquote=true}

\begin{document}
  \thispagestyle{empty}
  \pagestyle{empty}
  {\Large\bfseries Programación---Prueba solemne 1, viernes 30 de abril de 2010}

  Nombre: \nombre

  \begin{enumerate}
    \item\pond{15}
      Evalúe las siguentes expresiones.
      Si la expresión es correcta,
      indique el \textbf{resultado} y el \textbf{tipo} del resultado.
      Si la expresión tiene un error,
      indique el \textbf{tipo de error}.

      \begin{tabular}{|l|l|l|}\hline
        \textbf{Expresión} &
            \textbf{Resultado} &
            \textbf{Tipo} (o \textbf{error}) \\\hline\hline
        \fixin!2 + 3! &
            \fixin!5! &
            \fixin!int! \\\hline
        \fixin!4 / 0! &
            &
            \fixin!ZeroDivisionError! \\\hline\hline
        \fixin!5 + 3 * 2!                        && \\\hline
        \fixin!'5' + '3' * 2!                    && \\\hline
        \fixin!2 ** 10 == 1000 or 2 ** 7 == 100! && \\\hline
        \fixin!int("cuarenta")!                  && \\\hline
        \fixin!64/16 + 100/25!                   && \\\hline
        \fixin!200 + 19%!                         && \\\hline
        \fixin!3 < (1024 % 10) < 6!              && \\\hline
        \fixin!'six' * 'eight'!                  && \\\hline
        \fixin!float(-int('5') + int('10'))!     && \\\hline
        \fixin!abs(len('ocho') - len('cinco'))!  && \\\hline
      \end{tabular}

    \item\pond{15}
      Rutee el siguiente programa
      con la entrada \lstinline!n = 1024!,
      y determine qué es lo que hace el programa.
      En el ruteo, ponga sólo un valor por cada línea.

      \hspace{2em}
      \begin{minipage}{.4\textwidth}
        \lstinputlisting[frame=single,numbers=left,numberstyle=\small]{ruteo1.py}
        \vspace{5ex}
        ¿Qué hace el programa?

        \framebox[\textwidth]{\rule[12.0ex]{0pt}{0pt}}
      \end{minipage}
      \hspace{4em}
      \begin{minipage}{.5\textwidth}
        \begin{tabular}{| p{2.5em} || *{4}{p{2.5em} |}}\hline
          Línea & \lstinline!n! & \lstinline!r! & \lstinline!b! & \lstinline!d! \\\hline\hline
          \hfill\small 1 & \hfill 1024 & & & \\\hline
          \hfill\small 2 &&&& \\\hline
          \hfill\small 3 &&&& \\\hline
          &&&& \\\hline &&&& \\\hline &&&& \\\hline
          &&&& \\\hline &&&& \\\hline &&&& \\\hline
          &&&& \\\hline &&&& \\\hline &&&& \\\hline
          &&&& \\\hline &&&& \\\hline &&&& \\\hline
          &&&& \\\hline
        \end{tabular}
      \end{minipage}

      \newpage 
      Nombre: \nombre

      \item\pond{20}
        Desarrolle un programa cuya entrada sean 
        tres números enteros \fixin!a!, \fixin!b! y \fixin!c!,
        y su salida sean los números en orden de menor a mayor.

        Por ejemplo, si la entrada es
        \fixin!a = 5!, \fixin!b = 1! y \fixin!c = 8!,
        la salida debe ser: \verb!1 5 8!.

        Recuerde que puede usar las funciones
        \lstinline+min()+ y \lstinline+max()+.

        \framebox[\textwidth]{\rule[39ex]{0pt}{0pt}}

      \item\pond{25}
        La media armónica de una secuencia de $n$ números reales
        $x_1, x_2, \dots, x_n$ se define como:
        \[
          H = \frac{n}{
            \frac{1}{x_1} +
            \frac{1}{x_2} + \cdots +
            \frac{1}{x_n}}
        \]
        Desarrolle un programa que calcule la media armónica
        de una secuencia de números.

        El programa primero debe preguntar al usuario
        cuántos números ingresará.
        A continuación, pedirá al usuario que ingrese
        cada uno de los $n$ números reales.
        Finalmente, el programa mostrará el resultado.

        \framebox[\textwidth]{\rule[39ex]{0pt}{0pt}}
        
      \newpage 
      Nombre: \nombre

      \item\pond{25}
        El número $e\approx 2.71828$
        puede ser representado como la siguiente suma infinita:
        \[
          e =
          \frac{1}{0!} +
          \frac{1}{1!} +
          \frac{1}{2!} +
          \frac{1}{3!} +
          \frac{1}{4!} +
          \cdots
        \]

        Desarrolle un programa que entregue un valor aproximado de $e$,
        calculando esta suma hasta llegar a un sumando
        que sea menor que $0.0001$.

        Recuerde que el factorial $n!$
        es el producto de los números de $1$ a $n$.

        
        \framebox[\textwidth]{\rule[90ex]{0pt}{0pt}}
  \end{enumerate}
\end{document}

