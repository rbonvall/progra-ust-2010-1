\documentclass[10pt,spanish]{article}
\usepackage[utf8]{inputenc}
\usepackage{babel}
\usepackage{fullpage}
\usepackage{listings}
\usepackage{mathpazo}
\usepackage{courier}
\usepackage{xcolor}
\usepackage{textcomp}
\usepackage{amssymb}
\usepackage{tikz}

\newcommand{\onelinerule}{\rule[2.3ex]{0pt}{0pt}}
\newcommand{\twolinerule}{\rule[6.2ex]{0pt}{0pt}}
\newcommand{\respuesta}{\framebox[\textwidth]{\twolinerule}}
\newcommand{\nombre}{\framebox[0.8\textwidth]{\onelinerule}}
%\renewcommand{\arraystretch}{1.2}
\newcommand{\cv}{\texttt}

\newcommand{\card}[3]{%
  \draw #1 rectangle +(2, 4);
  %\draw +(0.5, 3.5) node {#2};
  %\draw +(0.5, 2.6) node {\ensuremath{#3}};
  \draw +(-1.5, -0.5) node {#2};
  \draw +(0.5, 2.6) node {\ensuremath{#3}};
}

\lstdefinelanguage{py3}{%
  classoffset=0,%
    morekeywords={%
      False,class,finally,is,return,None,continue,for,lambda,try,%
      True,def,from,nonlocal,while,and,del,global,not,with,%
      as,elif,if,or,yield,assert,else,import,pass,break,except,in,raise},%
    keywordstyle=\color{black!80}\bfseries,%
  classoffset=1,
    morekeywords={int,float,str,abs,len,input,print,exit,range,min,max,%
      set,dict,tuple,list},
    keywordstyle=\color{black!50}\bfseries,%
  classoffset=0,%
  sensitive=true,%
  morecomment=[l]\#,%
  morestring=[b]',%
  morestring=[b]",%
  stringstyle=\em,%
}
\lstset{language=py3}
\lstset{basicstyle=\ttfamily}
\lstset{columns=fixed}
\lstset{upquote=true}
\lstset{upquote=true}
\lstset{showstringspaces=false}
\listfiles

\begin{document}
  \thispagestyle{empty}
  \pagestyle{empty}
  {\Large\bfseries Programación---Prueba solemne 4, viernes 9 de julio de 2010}

  Nombre: \nombre

  \vspace{3ex}
  \begin{minipage}[t]{.6\textwidth}
    Un supermercado tiene la información de sus productos
    en un archivo llamado~\texttt{productos.txt}.
    Cada línea del archivo tiene tres datos:
    \begin{itemize}
      \item el código del producto (un número entero),
      \item el nombre del producto, y
      \item el precio del producto.
    \end{itemize}
  Los datos están separados por una coma (``\texttt{,}'').
  \end{minipage}
  \hfill
  \begin{minipage}[t]{.3\textwidth}
    \texttt{productos.txt}:
    \begin{lstlisting}[language={},frame=single]
1265,Reloj,5000
613,Cuaderno,900
9801,Vuvuzela,15000
321,Lapiz,400
5413,Martillo,3000
857,Paleta,6000
612,Pizza,1200
    \end{lstlisting}
  \end{minipage}

  \begin{enumerate}
    \item Escriba una función \lstinline!existe_producto(codigo)!
      que indique si existe en el archivo
      el producto con el código dado:
      \begin{lstlisting}
>>> existe_producto(321)
True
>>> existe_producto(512)
False
      \end{lstlisting}

    \item Escriba una función \lstinline!cuenta_menores_que(precio_maximo)!
      que entregue la cantidad de productos
      cuyo precio es menor que \lstinline!precio_maximo!:
      \begin{lstlisting}
>>> cuenta_menores_que(1000)
2
>>> cuenta_menores_que(7500)
6
>>> cuenta_menores_que(300)
0
      \end{lstlisting}
      \newpage

    \item
      \begin{minipage}[t]{.6\textwidth}
        A partir del archivo \texttt{productos.txt},
        escriba un programa que cree un nuevo archivo
        llamado \texttt{nombres.txt},
        que tenga sólo los nombres de los productos.

        \vspace{45ex}
      \end{minipage}
      \hfill
      \begin{minipage}[t]{.3\textwidth}
        \texttt{nombres.txt}:
        \begin{lstlisting}[language={},frame=single]
Reloj
Cuaderno
Vuvuzela
Lapiz
Martillo
Paleta
Pizza
        \end{lstlisting}
      \end{minipage}

    \item
      Escriba la salida de cada programa.

      \foreach \x in {1,2,3,4} {
        \noindent
        \begin{minipage}[b]{22.5em}
          \lstinputlisting{programas-prueba-4/p\x.py}
          \framebox[22em]{\rule[15ex]{0pt}{0pt}}
          \vspace{0.7em}
        \end{minipage}
      }

  \end{enumerate}
\end{document}

