\documentclass[10pt,spanish]{article}
\usepackage[utf8]{inputenc}
\usepackage{babel}
\usepackage{fullpage}
\usepackage{listings}
\usepackage{mathpazo}
\usepackage{courier}
\usepackage{xcolor}
\usepackage{textcomp}
\usepackage{amssymb}
\usepackage{tikz}

\newcommand{\onelinerule}{\rule[2.3ex]{0pt}{0pt}}
\newcommand{\twolinerule}{\rule[6.2ex]{0pt}{0pt}}
\newcommand{\respuesta}{\framebox[\textwidth]{\twolinerule}}
\newcommand{\nombre}{\framebox[0.9\textwidth]{\onelinerule}}
\newcommand{\li}{\lstinline}

\lstdefinelanguage{py3}{%
  classoffset=0,%
    morekeywords={%
      False,class,finally,is,return,None,continue,for,lambda,try,%
      True,def,from,nonlocal,while,and,del,global,not,with,%
      as,elif,if,or,yield,assert,else,import,pass,break,except,in,raise},%
    keywordstyle=\color{black!80}\bfseries,%
  classoffset=1,
    morekeywords={int,float,str,abs,len,input,print,exit,range,min,max,%
      set,dict,tuple,list},
    keywordstyle=\color{black!50}\bfseries,%
  classoffset=0,%
  sensitive=true,%
  morecomment=[l]\#,%
  morestring=[b]',%
  morestring=[b]",%
  stringstyle=\em,%
}
\lstset{language=py3}
\lstset{basicstyle=\ttfamily}
\lstset{columns=fixed}
\lstset{upquote=true}
\lstset{upquote=true}
\lstset{showstringspaces=false}

\begin{document}
  \thispagestyle{empty}
  \pagestyle{empty}
  {\Large\bfseries Programación---Examen, 30 de julio de 2010}

  Nombre: \nombre

  \vspace{1ex}
  \begin{enumerate}

    \item
      Escriba una función \li!digitos(n)!
      que entregue una lista con los dígitos de \li!n!:
      \begin{lstlisting}[gobble=8]
        >>> digitos(142857)
        [1, 4, 2, 8, 5, 7]
      \end{lstlisting}

    \vspace{48ex}
        
    \item
      Escriba una función \li!es_primo(n)!
      que indique si el número \li!n! es primo o no:
      \begin{lstlisting}[gobble=8]
        >>> es_primo(9)                  >>> es_primo(2)         
        False                            True                    
        >>> es_primo(17)                 >>> es_primo(1)         
        True                             False                   
      \end{lstlisting}

    \newpage

    \item
      Escriba una función \li!suma_cuadrados_primos(m)!
      que entregue la suma de los cuadrados de los números primos
      menores que \li!m!:
      \begin{lstlisting}[gobble=8]
        >>> suma_cuadrados_primos(10)
        87
        >>> 2 * 2 + 3 * 3 + 5 * 5 + 7 * 7
        87
      \end{lstlisting}

    \vspace{52ex}

    \item 
      Escriba una función \li!mayores_que_el_promedio(l)!
      que reciba una lista \li!l! de números reales,
      que entregue la lista de los valores
      que son mayores que el promedio,
      ordenados de menor a mayor:
      \begin{lstlisting}[gobble=8]
        >>> valores = [3.5, 2.0, 4.0, 1.0, 3.0]
        >>> mayores_que_el_promedio(valores)
        [3.0, 3.5, 4.0]
      \end{lstlisting}

    \newpage

    \item
      Las fechas de nacimiento de varias personas están guardadas en un diccionario.
      Cada llave es el nombre,
      y cada valor es una tupla de tres elementos:
      el año, el mes y el día:
      \begin{lstlisting}[gobble=8]
        fechas_de_nacimiento = {
          'Pepito': (1983, 5, 14),
          'Fulanita': (1985, 5, 7),
          'Jaimito': (1955, 3, 3),
          'Yayita': (1962, 12, 4),
          'Zutanita': (1986, 11, 24),
        }
      \end{lstlisting}

      \begin{enumerate}
        \item
          Escriba una función \li!cuenta_cumples_mes(m)!
          que cuente cuántas personas tienen cumpleaños en el mes \li!m!:
          \begin{lstlisting}[gobble=12]
            >>> cuenta_cumples_mes(5)
            2
            >>> cuenta_cumples_mes(8)
            0
          \end{lstlisting}

        \vspace{40ex}

        \item
          Escriba una función \li!edades(fechas_de_nacimiento)!
          que retorne un diccionario que asocie a cada persona
          su edad:
          \begin{lstlisting}[gobble=12]
            >>> edades(fechas_de_nacimiento)
            {'Zutanita': 24, 'Pepito': 27, 'Fulanita': 25, 'Jaimito': 55, 'Yayita': 48}
          \end{lstlisting}
        
      \end{enumerate}

  \end{enumerate}
\end{document}

